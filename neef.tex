\documentclass{article}
\title{National Environmental Education Foundation}
\author{Team members go here}

\begin{document}
\maketitle
\abstract{
We present suggestions as to how
NEEF may use ``data'' more effectively.
In particular, we focus on ways they can use data to 
ascertain how their initiatives impact the public
and target their campaigns.}

\section{Our approach}
We broke into three groups to investigate different
aspects of how data can be used more effectively.

* Existing Datasets
** Is there any existing data indicative of some of these actions?
** Project - Search for these, list their sources, description of one row of the data.

* Advertising Research
** How do advertising agencies measure reach?  Print ads / media ads.  1st order / 2nd order reach.
** How do advertising agencies measure action?  Who actually bought something they wouldn't have before because of an ad?
** How do we collect digital information that might answer these questions?  Facebook / Google ad models?

* Behavioral Analysis
** Were people more likely to do something after working with NEEF than before?
** How do you know?  (Baselines, control groups, A/B testing)
** How do you measure reach vs. actual action?

=== Others ===

* Standardize Data Collection
** What data or approaches are common across all programs?

* Technology for Data Collection
** Are there digital metrics that are correlated with (or directly indicative of) activity? 
** Google Analytics, Twitter campaigns, social media and social networks

\section{Measuring effectiveness}

\subsection{NEEF's Goals}
NEEF tries to reach the public to encourage direct actions that protect nature and improve the health of the public. Here are some examples of actions in which that they would like to see increases

* Hiking
* Conserving water
* Recycling
* Carpooling
* Public transit use
* Studies/careers in the environment

Moreover, they hope to instill a mindset of consideration of the environment.

\section{External data}
Cavan's suggestions
* Look the [http://thedata.org/ Dataverse] for national data
* The main environmental data would probably be on brownfields or air quality.
* The [http://www.bls.gov/tus/ American Time Use Survey] might tell us about activities related to the environment.


Desired Outcome/Measures:
Transportation
Census Bureau (Carpooling rate available in ACS Table S0802); Texas Transportation Institute; American Public Transportation Association
 
Public Lands Visits
 NPS - [http://www.nature.nps.gov/stats/park.cfm]
Can slice visits by state, park, etc
Compare before/after/on Public Lands Day


Water conservation
EPA
 
Farmers Markets
USDA
 
Environmental Attitudes/Knowledge
Pew [http://www.pewglobal.org/category/datasets/2010/]; Roper [http://www.ropercenter.uconn.edu/data_access/tag/global_warming.html#.T1JjXHluDT4] (All at national scale)
 
Waste diversion (recycling)
EPA [http://www.epa.gov/osw/nonhaz/municipal/msw99.htm](finest scale= Census Regions)

== Topics to analyze ==

\section{Findings}
\subsection{Recommendation for Health and Environment Program (Pediatrics)- Adrish Sannyasi}


Health and Environment Impact Measurement (Future State hypothesis)

Goal: Measure quality of care improvements through advanced environmental health knowledge. Focus on Institution level decision making (Kaiser, Mayo etc.)- pediatric medicine and nursing.

Actors: Primary Care physicians in the health system under measurement

Future State Process:

Primary Care --> Environment History + Clinical Data--> Environment Knowledge-> Selective Interventions-> Improve disease prognosis/clinical outcome

Implementation Plan:

1.	Partner with Integrated Delivery Systems: Kaiser, Intermountain, Geisinger, Mayo, Group Health or HHS Beacon community.

http://healthit.hhs.gov/portal/server.pt?open=512&objID=1805&parentname=CommunityPage&parentid=2&mode=2&cached=true

2.	Environment History to become part of digitized medical record
3.	Track the clinical interventions based on environmental history information
4.	Environmental Health knowledge through awareness and continuing medical education program.
5.	Measure intervention data to evaluate effectiveness of the education programs.

Incentives for Health Systems:
The cost for environmentally attributable childhood disease is $54.9 billion. This strategy is less expensive than current alternative.
\end{document}
