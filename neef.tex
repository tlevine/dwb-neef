\documentclass{article}
\title{National Environmental Education Foundation}
\author{Team members go here}

\begin{document}
\maketitle
\abstract{
We present suggestions as to how
NEEF may use ``data'' more effectively.
In particular, we focus on ways they can use data to 
ascertain how their initiatives impact the public
and target their campaigns.}

\section{Operationalizing NEEF's Goals}
NEEF tries to reach the public to encourage direct actions that protect nature and improve the health of the public.
They hope to instill a mindset of consideration of the environment, and such a mindset might lead to an increase
in the activities like the following.
\begin{itemize}
\item Hiking
\item Conserving water
\item Carpooling
\item Studies/careers in the environment
\end{itemize}

Data on these activities can be seen as operationalizations of the mindset of concern for the environment.

\section{Existing Datasets}

\subsection{External datasets}
We searched for existing datasets that could be used as measures.

We found several large surveys that
gather data on activities related to NEEF's mission.
NEEF may be able to use these as measures of
relevant human behavior in studies of the effect of their campaign.

Also, there are a lot of environmental data on brownfields or air quality.
These may not be very helpful for measuring how human behavior changes,
but they may help NEEF to target their campaigns.

\subsection{Internal datasets}
We identified datasets that NEEF already has and may be able to investigate further.

NEEF keeps records of grant recipients. These may contain insights regarding
how grants turned out.

Google Analytics already collects some information that you may not have noticed.
In particular, it already tracks clicks on links to external websites, so you
can see whether people went to other sites after seeing the NEEF site.

NEEF uses ConstantContact for its newsletter. ConstantContact may already
provide analytics about the readership of the newsletter.

\section{Behavioral Analysis}
Determining whether NEEF campaigns had an impact
can be easier if studies are designed in a particular way.
We developed some suggestions of stastical concerns that
NEEF might want to consider.

** Were people more likely to do something after working with NEEF than before?
** How do you know?  (Baselines, control groups, A/B testing)
** How do you measure reach vs. actual action?

\end{document}
